\chapter{Introduction to ImageJ}
\thispagestyle{fancy}
\fancyhead[RE,LO]{Technical Document \thechapter}
This year in PHY 2131 we will be using ImageJ for our labs. 
%You may have been exposed to this software through research, internships, or BSCI 205. 
This software is a great tool for use in scientific inquiry. 
Before we teach you how to use it, we think it is important for you to know a little about it. 
ImageJ is an image analysis software developed at the National Institutes of Health (NIH) in 1997. 
Its author, Wayne Rasband, originally wrote it for use by biomedical researchers working with microscope images. 
He designed it with an open architecture so that anyone could write `plugins' to add new tricks to the program. 
As a result, its capabilities are constantly growing and improving. 
Since then, this open-source, Java-based image processing program has become a standard tool for image analysis in many fields, including biology, medicine, radiology, microscopy, etc.
\par 
Depending on what plugins (added program code) are employed, ImageJ can:
\begin{itemize}
\item display, edit, analyze, process, save, and print 8-bit color and grayscale, 16-bit integer and 32-bit images
\item read many image formats, as well as videos and image ``stacks'' (a sequence of image frames sharing the same window)
\item measure distances and angles
\item do geometric transformations such as scaling, rotation, and flips
\item perform standard image processing functions such as logical and arithmetical operations between images, contrast manipulation, convolution, Fourier analysis, sharpening, smoothing, edge detection, and median filtering
\item calculate area and pixel value statistics of user-defined selections and intensity thresholded objects
\item create density histograms and line profile plots
\end{itemize}
Seeing is believing — so rather than engage in a dry discussion of how great ImageJ is and how useful it is in current fields of research, here are a few things you can and should do for yourself:
\begin{enumerate}
\item Visit this website: ``The Cell, an image library'' (\url{http://www.cellimagelibrary.org/}), and look at some of the images produced using ImageJ by scientists around the world. There are video files and still images. It is a searchable database with divisions for Cell Process, Component, Type, and Organism. (Run by the American Society for Cell Biology.)
\item If you are an image analysis/processing junkie, you may want to check out the PowerPoint ``ImageJ, A Useful Tool for Image Processing and Analysis'' (\url{rsb.info.nih.gov/ij/docs/examples/IJ-M&M08.ppt}) created by Joel Sheffield of Temple University. This has lots of information and can give you a feel for the program. We will not use all of these functions in our labs, but it gives you a feel for what ImageJ can do.
\item Check out the list below of some of the scientific poster presentations that were given at the recent ImageJ Conference in Luxembourg. Very cutting edge.
\end{enumerate}

\section*{Selected List of Scientific Poster Presentations at the recent ImageJ Conference, Oct. 2012, Luxembourg}
Links to learn more @ \url{http://imagejconf.tudor.lu/program/start}
\begin{itemize}
\item ``Segmentation and Tracking 4D of C.Elegans early embryogenesis,'' Jaza Gul Mohammed
\item ``Usage of ImageJ program for visualization and analysis of microarray experiments data,'' Denis V. Volkov
\item ``High-throughput Quantification and Analysis of T-Lymphocyte Killing Efficiency with ImageJ,'' Louis Wolf
\item ``Measurement of Nano-particle Uptake in Live Cells using ImageJ,'' Victoria Machtey
\item ``ImageJ in the workflow for generating, evaluating and visualizing 3D gene expression atlas,'' Albina Asadulina
\item ``An ImageJ macro to analyze mitochondrial movement along axon,'' Lai Ding
\item ``Assessment of the surface aspect of foods using ImageJ plugins,'' Lorenzo Fongaro
\item ``Morphometric test-system for patients with ischemic cardiomyopathy,'' Sergey Gutor
\item ``Ultrasonograhic Textural Pattern of Tendon: Analysis with Grey Level Co-occurrence Matrices,'' José Rios-Diaz
\item ``MRE-J: A Novel Pipeline For Magnetic Resonance Elastography Image Processing Using ImageJ and Apache Commons-Math,'' Eric Barnhill
\item ``Using ImageJ to assess radiographic and ultrasound digital images,'' Stefan Andrei
\end{itemize}